\documentclass[11pt,]{article}
\usepackage{lmodern}
\usepackage{amssymb,amsmath}
\usepackage{ifxetex,ifluatex}
\usepackage{fixltx2e} % provides \textsubscript
\ifnum 0\ifxetex 1\fi\ifluatex 1\fi=0 % if pdftex
  \usepackage[T1]{fontenc}
  \usepackage[utf8]{inputenc}
\else % if luatex or xelatex
  \ifxetex
    \usepackage{mathspec}
  \else
    \usepackage{fontspec}
  \fi
  \defaultfontfeatures{Ligatures=TeX,Scale=MatchLowercase}
\fi
% use upquote if available, for straight quotes in verbatim environments
\IfFileExists{upquote.sty}{\usepackage{upquote}}{}
% use microtype if available
\IfFileExists{microtype.sty}{%
\usepackage{microtype}
\UseMicrotypeSet[protrusion]{basicmath} % disable protrusion for tt fonts
}{}
\usepackage[margin=1in]{geometry}
\usepackage{hyperref}
\hypersetup{unicode=true,
            pdftitle={Insecticide rotations delay evolution of resistance in a minority of model runs when compared to sequential use.},
            pdfborder={0 0 0},
            breaklinks=true}
\urlstyle{same}  % don't use monospace font for urls
\usepackage{longtable,booktabs}
\usepackage{graphicx,grffile}
\makeatletter
\def\maxwidth{\ifdim\Gin@nat@width>\linewidth\linewidth\else\Gin@nat@width\fi}
\def\maxheight{\ifdim\Gin@nat@height>\textheight\textheight\else\Gin@nat@height\fi}
\makeatother
% Scale images if necessary, so that they will not overflow the page
% margins by default, and it is still possible to overwrite the defaults
% using explicit options in \includegraphics[width, height, ...]{}
\setkeys{Gin}{width=\maxwidth,height=\maxheight,keepaspectratio}
\IfFileExists{parskip.sty}{%
\usepackage{parskip}
}{% else
\setlength{\parindent}{0pt}
\setlength{\parskip}{6pt plus 2pt minus 1pt}
}
\setlength{\emergencystretch}{3em}  % prevent overfull lines
\providecommand{\tightlist}{%
  \setlength{\itemsep}{0pt}\setlength{\parskip}{0pt}}
\setcounter{secnumdepth}{0}
% Redefines (sub)paragraphs to behave more like sections
\ifx\paragraph\undefined\else
\let\oldparagraph\paragraph
\renewcommand{\paragraph}[1]{\oldparagraph{#1}\mbox{}}
\fi
\ifx\subparagraph\undefined\else
\let\oldsubparagraph\subparagraph
\renewcommand{\subparagraph}[1]{\oldsubparagraph{#1}\mbox{}}
\fi

%%% Use protect on footnotes to avoid problems with footnotes in titles
\let\rmarkdownfootnote\footnote%
\def\footnote{\protect\rmarkdownfootnote}

%%% Change title format to be more compact
\usepackage{titling}

% Create subtitle command for use in maketitle
\newcommand{\subtitle}[1]{
  \posttitle{
    \begin{center}\large#1\end{center}
    }
}

\setlength{\droptitle}{-2em}
  \title{Insecticide rotations delay evolution of resistance in a minority of
model runs when compared to sequential use.}
  \pretitle{\vspace{\droptitle}\centering\huge}
  \posttitle{\par}
  \author{}
  \preauthor{}\postauthor{}
  \date{}
  \predate{}\postdate{}

\usepackage{setspace}
\doublespacing
\usepackage{lineno}
\linenumbers

\begin{document}
\maketitle

Andy South * and Ian M. Hastings

Department of Parasitology, Liverpool School of Tropical Medicine,
Liverpool L3 5QA, UK.

\href{mailto:southandy@gmail.com}{\nolinkurl{southandy@gmail.com}},
\href{mailto:andy.south@lstmed.ac.uk}{\nolinkurl{andy.south@lstmed.ac.uk}}

\href{mailto:ian.hastings@lstmed.ac.uk}{\nolinkurl{ian.hastings@lstmed.ac.uk}}

* Corresponding author

\subsection{Abstract}\label{abstract}

Insecticide resistance threatens the control of the vectors of dangerous
diseases including malaria, dengue and zika. Recent increases in
insecticide resistance in public health are an evolutionary process
caused by sustained exposure of insect populations to a small number of
available insecticides. Resistance is a particular problem in public
health as compared to agriculture because insecticides need to be longer
lasting to provide affordable protection and there are fewer available
active ingredients. Efforts to limit the development of insecticide
resistance are grouped under the term Insecticide Resistance Management
(IRM). The main strategies advocated to reduce the development of
resistance are 1) rotations, 2) sequences, 3) mixtures and 4) mosaics.
Rotations regularly switch between the use of different insecticides
with a short time step (one or a few years) irrespective of resistance
levels. Sequences, in contrast, switch from one insecticide to another
only when resistance levels have reached a critical, defined threshold
usually over a longer timescale.

Rotations are often advocated as one of the best options for limiting
the development of resistance.

Testing different IRM strategies in the field is difficult and there has
been little recent modelling work on insecticide rotations.

In this paper a model is described allowing rotations and sequences to
be compared in terms of their effect on the evolution of insecticide
resistance. The model is used to develop a mechanistic understanding of
when and why insecticide resistance is likely to evolve faster with
rotations or sequences.

The results suggest that under more likely circumstances than not the
evolution of insecticide resistance will reach resistance thresholds at
very similar times for rotations and sequences.

However under a less common, but still plausible, set of circumstances
there are predicted to be large adavnatages to using a rotations as
compared to a sequence.

The advantages to a rotation only occur when there are costs of
resistance or dispersal from untreated areas, and there is high
dominance of cost and low dominance of selection.

The mechanisms for these results are explained.

Other operational factors will favour rotations or sequences aside from
the implications for the evolution of resistance. Developing an
understanding of the evolutionary implications allows a more explicit
consideration of these other factors on their own merits.

\pagebreak

\subsection{Introduction}\label{introduction}

Despite much field and modelling work on the evolution of resistance
under different insecticide or drug strategies {[}1{]} there is a small
evidence base to support decisions about the use of insecticides in
public health {[}2{]}.

Rotations are advocated as one of the most favoured approaches for
Insecticide Resistance Management (IRM){[}3{]}.

In a rotation the insecticide used is changed on a set time interval
irrespective of resistance levels. Sequential use, in contrast, denotes
the practice of using an insecticide until resistance to it reaches a
certain threshold and then switching to another. Different rotation time
intervals can be considered, e.g.~in vector control for Malaria, Indoor
Residual Spraying may be repeated every year (around 10 anopheles
mosquito generations) and Insecticide Treated Nets may be replaced every
two years (around 20 generations). Multiples of these intervals could be
used as the rotation interval. Multiple generational rotations like this
contrast with the situation in agriculture where a rotation is sometimes
more tightly defined as having an interval of a single generation
{[}4{]}.

Here a modern modelling approach is described to allow an assessment of
the potential benefits, in terms of the evolution of resistance, of
rotations over sequences.

\subsection{Methods}\label{methods}

A population genetic model was developed tracking the change in
resistance allele frequency as consequence of the genetic determination
of the resistance phenotype, and the deployment policy for the
insecticides. The model uses a similar approach to that previously used
to compare insecticide mixtures and sequences {[}5{]}. The
implementation is simpler here because for rotations and sequences there
is only a need to follow one insecticide at a time and linkage
disequilibrium can be ignored. The simpler implementation allows the
model to be run for an unlimited number of insecticides. The algebra
behind the model is described in the supplementary information.

\begin{enumerate}
\def\labelenumi{\arabic{enumi}.}
\item
  Sequences : one insecticide is used until resistance threshold
  frequency (0.5 in this case) is reached, then switch to a new
  insecticide.
\item
  Rotations : use insecticide for set time (i.e.~number of insect
  generations), switch to another.
\end{enumerate}

In both cases the choice of the new insecticide to use was made by going
through the list of available insecticides and choosing the next one for
which resistance was below the threshold. Thus there is the potential to
switch back to insecticides that had been used earlier in a model run if
resistance had not reached the threshold or if resistance had declined
below the threshold during a period when not in use. This means that the
``sequential'' policy is effectively a ``rotate when resistant'' policy
but we use the former term for consistency with previous work.
\emph{todo cite different terms from Rex consortium review} For both
rotations and sequences the simulation continues until no more
insecticides remain below their resistance thresholds.

Within the model inputs effect fitness as shown in Figure 1, differences
in fitness lead to the changes in allele frequencies over time.

\textbf{Table 1. Model inputs and the ranges used in simulations}

\begin{longtable}[]{@{}llll@{}}
\toprule
\begin{minipage}[b]{0.28\columnwidth}\raggedright\strut
Input\strut
\end{minipage} & \begin{minipage}[b]{0.46\columnwidth}\raggedright\strut
Description\strut
\end{minipage} & \begin{minipage}[b]{0.07\columnwidth}\raggedright\strut
Min\strut
\end{minipage} & \begin{minipage}[b]{0.07\columnwidth}\raggedright\strut
Max\strut
\end{minipage}\tabularnewline
\midrule
\endhead
\begin{minipage}[t]{0.28\columnwidth}\raggedright\strut
Effectiveness\strut
\end{minipage} & \begin{minipage}[t]{0.46\columnwidth}\raggedright\strut
proportion of susceptible (SS) insects killed by exposure to
insecticide\strut
\end{minipage} & \begin{minipage}[t]{0.07\columnwidth}\raggedright\strut
0.5\strut
\end{minipage} & \begin{minipage}[t]{0.07\columnwidth}\raggedright\strut
1\strut
\end{minipage}\tabularnewline
\begin{minipage}[t]{0.28\columnwidth}\raggedright\strut
Exposure\strut
\end{minipage} & \begin{minipage}[t]{0.46\columnwidth}\raggedright\strut
proportion of females exposed to insecticide\strut
\end{minipage} & \begin{minipage}[t]{0.07\columnwidth}\raggedright\strut
0.4\strut
\end{minipage} & \begin{minipage}[t]{0.07\columnwidth}\raggedright\strut
0.9\strut
\end{minipage}\tabularnewline
\begin{minipage}[t]{0.28\columnwidth}\raggedright\strut
Male exposure\strut
\end{minipage} & \begin{minipage}[t]{0.46\columnwidth}\raggedright\strut
proportion of males exposed to insecticide as proportion of
females\strut
\end{minipage} & \begin{minipage}[t]{0.07\columnwidth}\raggedright\strut
0\strut
\end{minipage} & \begin{minipage}[t]{0.07\columnwidth}\raggedright\strut
1\strut
\end{minipage}\tabularnewline
\begin{minipage}[t]{0.28\columnwidth}\raggedright\strut
Resistance restoration\strut
\end{minipage} & \begin{minipage}[t]{0.46\columnwidth}\raggedright\strut
ability of resistance (RR) to restore fitness of insects exposed to
insecticide\strut
\end{minipage} & \begin{minipage}[t]{0.07\columnwidth}\raggedright\strut
0.1\strut
\end{minipage} & \begin{minipage}[t]{0.07\columnwidth}\raggedright\strut
0.9\strut
\end{minipage}\tabularnewline
\begin{minipage}[t]{0.28\columnwidth}\raggedright\strut
Dominance of resistance\strut
\end{minipage} & \begin{minipage}[t]{0.46\columnwidth}\raggedright\strut
sets fitness of heterozygous (SR) insects between that of SS \& RR in
presence of insecticide\strut
\end{minipage} & \begin{minipage}[t]{0.07\columnwidth}\raggedright\strut
\strut
\end{minipage}\tabularnewline
\begin{minipage}[t]{0.28\columnwidth}\raggedright\strut
Frequency\strut
\end{minipage} & \begin{minipage}[t]{0.46\columnwidth}\raggedright\strut
frequency of resistance alleles within the population\strut
\end{minipage} & \begin{minipage}[t]{0.07\columnwidth}\raggedright\strut
0.005\strut
\end{minipage} & \begin{minipage}[t]{0.07\columnwidth}\raggedright\strut
0.1\strut
\end{minipage}\tabularnewline
\begin{minipage}[t]{0.28\columnwidth}\raggedright\strut
Cost of resistance\strut
\end{minipage} & \begin{minipage}[t]{0.46\columnwidth}\raggedright\strut
fitness of resistant (RR) insects in absence of insecticide\strut
\end{minipage} & \begin{minipage}[t]{0.07\columnwidth}\raggedright\strut
0\strut
\end{minipage} & \begin{minipage}[t]{0.07\columnwidth}\raggedright\strut
0.1\strut
\end{minipage}\tabularnewline
\begin{minipage}[t]{0.28\columnwidth}\raggedright\strut
Dominance of cost\strut
\end{minipage} & \begin{minipage}[t]{0.46\columnwidth}\raggedright\strut
sets fitness of heterozygous (SR) insects between that of SS \& RR in
absence of insecticide\strut
\end{minipage} & \begin{minipage}[t]{0.07\columnwidth}\raggedright\strut
0.1\strut
\end{minipage} & \begin{minipage}[t]{0.07\columnwidth}\raggedright\strut
0.9\strut
\end{minipage}\tabularnewline
\begin{minipage}[t]{0.28\columnwidth}\raggedright\strut
Insecticide number\strut
\end{minipage} & \begin{minipage}[t]{0.46\columnwidth}\raggedright\strut
number of independent insecticides available\strut
\end{minipage} & \begin{minipage}[t]{0.07\columnwidth}\raggedright\strut
2\strut
\end{minipage} & \begin{minipage}[t]{0.07\columnwidth}\raggedright\strut
5\strut
\end{minipage}\tabularnewline
\begin{minipage}[t]{0.28\columnwidth}\raggedright\strut
Rotation interval\strut
\end{minipage} & \begin{minipage}[t]{0.46\columnwidth}\raggedright\strut
number of generations to use insecticide before switching\strut
\end{minipage} & \begin{minipage}[t]{0.07\columnwidth}\raggedright\strut
5\strut
\end{minipage} & \begin{minipage}[t]{0.07\columnwidth}\raggedright\strut
50\strut
\end{minipage}\tabularnewline
\begin{minipage}[t]{0.28\columnwidth}\raggedright\strut
Coverage\strut
\end{minipage} & \begin{minipage}[t]{0.46\columnwidth}\raggedright\strut
for scenarios with dispersal, the proportion of insects in the treated
area rather than untreated refugia\strut
\end{minipage} & \begin{minipage}[t]{0.07\columnwidth}\raggedright\strut
0.1\strut
\end{minipage} & \begin{minipage}[t]{0.07\columnwidth}\raggedright\strut
0.9\strut
\end{minipage}\tabularnewline
\begin{minipage}[t]{0.28\columnwidth}\raggedright\strut
Dispersal\strut
\end{minipage} & \begin{minipage}[t]{0.46\columnwidth}\raggedright\strut
proportion of population exchanged between treated and untreated areas
per generation, 0=none, 1=random mixing\strut
\end{minipage} & \begin{minipage}[t]{0.07\columnwidth}\raggedright\strut
0.1\strut
\end{minipage} & \begin{minipage}[t]{0.07\columnwidth}\raggedright\strut
0.9\strut
\end{minipage}\tabularnewline
\bottomrule
\end{longtable}

The model was run under four main scenarios :

\begin{enumerate}
\def\labelenumi{\arabic{enumi}.}
\tightlist
\item
  baseline : no resistance fitness costs or dispersal to/from untreated
  refugia
\item
  fitness costs of resistance added
\item
  dispersal link to untreated areas (no costs)
\item
  costs and dispersal
\end{enumerate}

For each scenario 10,000 runs were performed randomly selecting inputs
according to the ranges specified in Table 1 using a uniform
distribution in all cases.

\paragraph{Assessing relative performance of insecticide use
strategies}\label{assessing-relative-performance-of-insecticide-use-strategies}

\emph{todo change this bit} It is necessary to choose a measure to
quantify the performance of an insecticide-use strategy and enable
comparison with an alternative. For this we summed the number of
generations when an insecticide was deployed and the resistance allele
frequency for that insecticide was below 0.5. This allowed comparison
between strategies using a variable number of insecticides. Previously
time-to-resistance, namely the number of generations it takes to reach a
defined resistance threshold (usually a resistance allele frequency of
0.5), was used. Using time-to-resistance worked when considering just
two insecticides, but when the number of insecticides is a variable it
is not sufficient.

\subsection{Results}\label{results}

\paragraph{Results1. form of resistance evolution with and without costs
and
dispersal.}\label{results1.-form-of-resistance-evolution-with-and-without-costs-and-dispersal.}

Example model runs are shown with no costs or dispersal (Fig 2), with
just resistance costs (Fig 3) and with just dispersal from untreated
refugia (Fig 4).

When there are no resistance fitness costs or dispersal links to
untreated refugia resistance frequencies do not decline when an
insecticide is not in use (Fig 2). Thus for sequences once resistance
thresholds are reached the resistance frequencies remain at that level
and the insecticide cannot be re-used. For rotations resistance
frequencies step upwards when insecticides are in use and remain at a
plateau when they are not. If the rotation interval is short enough then
the insecticide can be used a few times before the resistance thresholds
are reached.

\emph{todo continue describing the example runs first, before talking
about overall results}

\subsubsection{Results2 : comparing rotations and sequences across model
runs.}\label{results2-comparing-rotations-and-sequences-across-model-runs.}

In all the example runs shown the resistance threshold (allele frequency
of 0.5) was reached for all insecticides within the course 500
generations which was the maximum limit imposed in the model. There were
many model runs {[}\emph{quantify later when run finalised}{]} in which
resistance frequencies for at least some insecticides were below
thresholds at the end of 500 generations. Such a run could be considered
a successful strategy given the conditions of the scenario because
insecticides below the resistance threshold are still available to be
used.

Thus to aid the comparison of rotations and sequences model runs can be
broadly classified into four groups : 1. both strategies succeeded : for
both rotations and sequences resistance thresholds for all insecticides
not reached within 500 generations. 2. neither strategy succeeded 3.
rotation only succeeded 4. sequence only succeeded

The number of model runs in each of these groups for the different
scenarios are shown in Fig 5.

\emph{todo : change the \%s because `succeeded' runs now left in} When
there were no costs or dispersal there was no difference between
rotations and sequences in the number of generations below the
resistance threshold (Fig 6A). With fitness costs included 67\% of model
runs, similar to those in the previous section, gave no difference
between rotations and sequences in the number of generations below the
resistance threshold. In contrast, 7\% of model runs generated an
advantage of greater than 20\% for rotations over sequences (Fig 6B).
Dispersal from untreated refugia gives smaller benefits of rotations
over sequences and in fewer runs (Fig 6C). With both costs and dispersal
the benefits of rotations are similar to when runs have costs alone (Fig
6D).

\paragraph{Runs where rotations have substantial advantage over
sequences}\label{runs-where-rotations-have-substantial-advantage-over-sequences}

\emph{Talk about classing runs into both `succeeded', both `failed' or
just rotation succeeded. }show violin plots here ?*

As shown in Fig 4.2 some runs with costs of resistance can show
substantial advantages for rotations over sequences. Examples of one
such run is shown in Fig 5.

A Partial Rank Correlation analysis identified that of the inputs,
dominance of cost and dominance of selection had the greatest influence
on the difference between rotations and sequences when resistance costs
were included (Fig 6). A combination of high values of dominance of cost
and low values for dominance of selection are required to give the
greatest advantages to rotations, but do not guarantee this advantage
(Fig 7).

\subsection{Discussion}\label{discussion}

Comparing time below resistance thresholds for rotations and sequences,
the model shows :

\begin{enumerate}
\def\labelenumi{\arabic{enumi}.}
\tightlist
\item
  rotations and sequences the same when no costs or dispersal
\item
  with resistance costs, 67\% of runs the same for rotations and
  sequences, but 6-7\% of runs have \textgreater{} 20\% advantage for
  rotations
\item
  with dispersal from untreated refugia but no costs, small advantage of
  rotations but very few runs \textless{} 0.1\% have \textgreater{}20\%
  advantage.
\end{enumerate}

This can be understood intuitively. When there are no costs, the
dynamics are the same, it is simply that sequential deployment proceeds
in a few, larger steps while rotations proceeds in more numerous smaller
steps; both reach the same point in around the same time (e.g.~figs 1
and 3). However dominance effects on cost and resistance can act to keep
frequencies low in a rotations policy. Recall that at low frequencies of
resistance most resistance alleles are present as heterozygotes RS and
only a vey small proportions heterozygotes. However costs are dominant
and still act at low frequencies. This is illustrated on Figure Y.
Relatively rapid rotations keep the allele frequencies at the LH side of
Figure Y where costs during the periods when the insecticide is not
deployed greatly outweigh the selection for IR during the periods when
the insecticide is deployed. We believe the explanation is therefore
that rotations keep allele frequencies in the parameter space to the
left of Figure Y and hence acts to slow the evolution of IR compared to
sequences which allow frequencies to rise to relatively high levels
before the insecticide is replaced. (does this sound reasonable??). Note
also that starting frequency, as expected, had an impact (Figure 7),
lower starting frequencies slightly favouring rotations by allowing more
time in the ``low frequency'' parameter space.

In rare circumstances, when there are costs of resistance, rotations can
lead to the resistance frequencies for a group of insecticides being
kept below resistance thresholds for the length of the simulations where
sequences run out of insecticides within around 100 generations. Is it
possible that these rare circumstances where parameter values are
`just-right' could be implemented operationally ? The problem is that
these rare circumstances rely on high costs of resistance, high
dominance of these costs and low dominance of the selection for the
resistance itself. These parameters are all not easy to measure and
there isn't agreement on their likely values {[}7{]}.

Note however that this combination of recessive resistance and high
dominance costs are exactly the features most desirable in any
insecticide irrespective of how it is deployed. We discuss fitness below
but note here that it is very difficult to maintain insecticide
concentration at the level required to make resistance recessive.
Concentrations decay post-application in most deployment strategies and
this cause resistance to gradually change from being recessive to being
dominant; see Figure 1 of {[}5{]} and our more extensive discussion in
{[}cite South et al WoS manuscript{]}.

The importance of resistance costs for substantial evolutionary benefits
of rotations over sequences points to the importance of the debate over
the fitness costs of insecticide resistance e.g. {[}7{]}. \emph{extend
this discussion of fitness costs}

We also found frequency of rotations had little effect. This is
important in the control of vectors of human diseases such as malaria
and dengue. These predominantly occur in resource-poor areas often with
poor healthcare infrastructure. Experience suggests that rotation
targets would often be missed (e.g rotations occur approximately every
15 generations rather than the planned ``official'' 10 generations) but
our simulations suggest this would not fatally undermine a rotations
policy.

\subsubsection{Caveats}\label{caveats}

The model does not include the potential effects of modifier genes
ameliorating the costs of resistance alleles. To include this would
require a more detailed model tracking linkage disequilibrium between
each resistance locus and its modifier. It has been suggested that
rotations could be favoured as a strategy because they keep resistance
frequences lower and restrict selection pressure for modifier genes
{[}3{]}. However, there is little evidence to support the existence of
such modifier genes in insectide resistance {[}{]}. Even with the
existence of modifier genes it is unlikely that they would provide much
of an advantage to rotations in the scenarios we describe. In the
modelled scenarios the use of an insecticide is stopped when the
resistance frequency for that insecticide reaches 0.5. Thus in all
scenarios resistance frequencies do not remain at high levels for long.
In this situation it seems unlikely that modifiers would be selected
for.

\subsubsection{Conclusions}\label{conclusions}

In their effect on the evolution of insecticide resistance, rotations
and sequences are most often the same. If costs of resistance are
included, infrequently (less than 7\% of model runs set between
plausible limits) there are substantial benefits to the rotation
strategy over a sequence. There was never a substantial benefit of a
sequence over a rotation. Whether or not insecticide rotations are
predicted to offer sizeable benefits over sequences is dependent on
coarse and fine scale issues about the nature of insecticide resistance
that are yet to be agreed upon. The model shows that resistance costs
are necessary to provide a substantial evolutionary advantage to
rotations. However even when these requirements are satisfied particular
combinations of inputs are required to generate the advantage of
rotations.

The model only compares rotations and sequences in terms of their effect
on the evolution of resistance. There are other operational reasons for
why a sequence or a rotation may be favoured.

In summary, we show that rotations are invariably better than sequential
use, although the difference may often be small (figure 4); certainly
rotations should not be seen as a panacea capable of removing the threat
of resistance evolution. Rotations may have a large impact if fitness
costs are prints and dominant (these are genetic/physiological factors
outside our control) and if resistance can be kept recessive. The latter
is under our control to some extent (fig 1 of Levick et al ) but it is
extremely difficult to deploy and maintain insecticide concentrations at
levels that ensure recessively (cite WoS ms): as concentrations decline
resistance become dominant and any befits of rotations largely
disappear.

\subsection{TODO}\label{todo}

\emph{add to methods description of how fitness determined in model,
move \& modify fig 1} \emph{try to come up with brief name for model
output to make it easier to refer to it later} \emph{add caveat about
polygenic resistance} \emph{The caption to each graph needs to have the
list of parameter values used to generate them} \emph{work on Ians fig Y
to help explanation}

\pagebreak

\subsection{Figures}\label{figures}

\begin{figure}

\includegraphics{pap3figs/Fig1-1} \hfill{}

\caption{Figure 1. [todo need to modify caption from MJ paper, start by saying that only in left panel during application of the insecticide] The effect of model inputs on the fitness of genotypes for a single insecticide. Fitness is shown on the y-axis and the different genotypes (SS, SR, RR) on the x axis. Firstly the exposure input determines the proportion of the population in the left and right panels (exposed and not exposed). For those that are exposed (left panel) insecticide effectiveness sets the fitness for SS, resistance restoration 'restores' a portion of the fitness for RR and dominance of resistance determines how the fitness for SR lies between that of SS and RR. For those that are not exposed, fitness of SS is set to 1 by definition, resistance cost determines the fitness of RR and again dominance of cost determines how the fitness for SR sits between that of SS and RR. In this example effectiveness=0.8, resistance restoration=0.5 which 'restores' half of the fitness lost due to the insecticide, dominance of resistance=0.7 which sets the fitness of the SR closer to RR than SS. Resistance cost=0.3 which reduces fitness in the absence of the insecticide from 1 to 0.7, and dominance of cost=0.8 which sets fitness of SR close to RR.}\label{fig:Fig1}
\end{figure}

\pagebreak

\begin{figure}

{\centering \includegraphics{pap3figs/Fig2-1} 

}

\caption{Figure 2. Comparing the increase in resistance frequencies over time for a rotation and a sequence when there no resistance fitness costs or dispersal from untreated refugia. The upper plot shows a rotation with a regular interval of 10 generations and the lower plot a sequence in which the insecticide is changed when a resistance threshold of 0.5 is reached. In this example the rotation and the sequence reach the endpoint, with all insecticides having a resistance frequency above 0.5, at the same time. For the sequence the resistance frequency for each insecticide increases to reach the threshold in one step and that insecticide cannot be used again. For the rotation the resistance frequency for each insecticide increases in shorter steps while it is in use and remains at the constant level while not. There are three insecticides, the resistance frequency for each is shown in the sub-plots, and the shaded boxes within each sub-plot show when that insecticide was in use. All other inputs are kept constant between the rotation and the sequence. Effectiveness=0.6, exposure=0.6, male exposure proportion=0.5, resistance restoration=0.5, dominance of resistance=1.}\label{fig:Fig2}
\end{figure}

\pagebreak

\begin{figure}

{\centering \includegraphics{pap3figs/Fig3-1} 

}

\caption{Figure 3. Comparing the increase in resistance frequencies over time for a rotation and a sequence when there are resistance fitness costs. The upper plot shows a rotation with a regular interval and the lower plot a sequence when the insecticide is changed when a resistance threshold of 0.5 is reached. In this example the rotation lasts longer than the sequence. For the sequence the resistance frequency for each insecticide increases to reach the threshold in one step but in contrast to the previous figure resistance frequencies decline when the insecticide is not in use allowing it to be used again. For the rotation the resistance frequency for each insecticide increases in shorter steps while it is in use and similarly declines  while not. There are three insecticides, the resistance frequency for each is shown in the sub-plots, and the shaded boxes within each sub-plot show when that insecticide was in use. The simulation is stopped when there are no remaining insecticides for which the resistance frequency is below the threshold. All other inputs are kept constant between the rotation and the sequence. Effectiveness=0.6, exposure=0.6, male exposure proportion=0.5, resistance restoration=0.5, dominance of resistance=1, cost=0.05, dominance of cost=1.}\label{fig:Fig3}
\end{figure}

\pagebreak

\begin{figure}

{\centering \includegraphics{pap3figs/Fig4-1} 

}

\caption{Figure 4. Comparing the increase in resistance frequencies over time for a rotation and a sequence with dispersal from untreated refugia. The upper plot shows a rotation with a regular interval and the lower plot a sequence when the insecticide is changed when a resistance threshold of 0.5 is reached. In this example the rotation and the sequence last the same time. In both cases the resistance frequencies in the treated areas (red lines) increase when an insecticide is in use. Resistance frequencies in the untreated areas (blue lines) increase after those in the treated areas as a result of dispersal from the treated areas. When an insecticide stops being used the resistance frequency in the treated area declines due to dispersal from the untreated area. For the sequence the resistance frequency for each insecticide increases to reach the threshold in one step and consistent with the previous figure resistance frequencies decline when the insecticide is not in use allowing it to be used again. For the rotation the resistance frequency for each insecticide increases in shorter steps while it is in use and similarly declines while not. There are three insecticides, the resistance frequency for each is shown in the sub-plots, and the shaded boxes within each sub-plot show when that insecticide was in use. The simulation is stopped when there are no remaining insecticides for which the resistance frequency is below the threshold. All other inputs are kept constant between the rotation and the sequence. Effectiveness=0.6, exposure=0.6, male exposure proportion=0.5, resistance restoration=0.5, dominance of resistance=1, cost=0, coverage=0.5, dispersal=0.1.}\label{fig:Fig4}
\end{figure}

\pagebreak

\begin{figure}

{\centering \includegraphics{pap3figs/Fig5-1} 

}

\caption{Figure 5. Success of model runs under different scenarios and strategies. Model runs classed as a success if there are still insecticides for which resistance is below the threshold of 0.5 after 500 generations (circa 50 years for Anopheles).}\label{fig:Fig5}
\end{figure}

\pagebreak

\textbackslash{}begin\{figure\}

\{\centering \includegraphics{pap3figs/Fig6-1}

\}

\textbackslash{}caption\{Figure 6. Effect of cost and dispersal on the
difference between rotations and sequences from model runs. The y axis
shows the number of model runs for the different caegories of difference
between rotations and sequences. A) when there are no resistance costs
or dispersal from untreated refugia, rotations and sequences produce the
same results. B),D) when costs are added there are infrequent model runs
where rotations last longer below resistance thresholds than sequences.
For between 6 and 7\% of runs including costs, rotations last more than
20\% longer than sequences. C) when dispersal from untreated refugia is
included without costs there are even less frequent runs where rotations
last longer than sequences (in just 4 out of 10000 runs rotations lasted
greater than 20\% longer than sequences).\}\label{fig:Fig6}
\textbackslash{}end\{figure\}

\pagebreak

\begin{figure}

{\centering \includegraphics{pap3figs/Fig7-1} 

}

\caption{Figure 7. Example of a model run where there is a substantial benefit of rotations over a sequence. For the sequence resistance thresholds are reached for all insecticides in a little over 100 generations or 10 years. For the rotation resistance levels for all insecticides are maintained well below resistance thresholds for the length of the simulation (500 generations or 50 years). Effectiveness=0.57, exposure=0.88, male exposure proportion=0.72, resistance restoration=0.68, dominance of resistance=0.04, cost=0.07, dominance of cost=0.59.}\label{fig:Fig7}
\end{figure}

\pagebreak

\begin{figure}

{\centering \includegraphics{pap3figs/Fig8-1} 

}

\caption{Figure 8. Which model inputs have the greatest influence on the difference between rotations and sequences ? In this case for scenario 2 where resistance costs are included but dispersal is not. Measured using Partial Rank Correlation Coefficients (PRCC). Positive values indicate a positive influence on the benefit of rotations, only cost and dominance of cost in this case. Dominance of cost and dominance of selection have the greatest magnitude of influence.}\label{fig:Fig8}
\end{figure}

\pagebreak

\textbackslash{}begin\{figure\}

\{\centering \includegraphics{pap3figs/Fig9-1}

\}

\textbackslash{}caption\{Figure 9. Model runs in which rotations perform
more than 20\% better than sequences are associated with high dominance
of cost and low dominance of selection. Dominance inputs are shown on
the x and y axes. Small grey points indicate all model runs. Model runs
in which rotations last greater than 20\% longer than sequences are
shown in colour with darker blue indicating those that have the greatest
difference.\}\label{fig:Fig9} \textbackslash{}end\{figure\}

\pagebreak

\subsection{References}\label{references}

\hypertarget{refs}{}
\hypertarget{ref-Consortium2013}{}
1. Consortium R. Heterogeneity of selection and the evolution of
resistance. Trends in Ecology \& Evolution. 2013;28:110--8.
doi:\href{https://doi.org/10.1016/j.tree.2012.09.001}{10.1016/j.tree.2012.09.001}.

\hypertarget{ref-Sternberg2017}{}
2. Sternberg ED, Thomas MB. Insights from agriculture for the management
of insecticide resistance in disease vectors. Evolutionary Applications.
2017; February:1--11.

\hypertarget{ref-WHO2012}{}
3. WHO. Global plan for insecticide resistance management in malaria
vectors (GPIRM). Geneva.; 2012.

\hypertarget{ref-Sudo2017}{}
4. Sudo M, Takahashi D, Andow DA, Suzuki Y, Yamanaka T. Optimal
management strategy of insecticide resistance under various insect life
histories: Heterogeneous timing of selection and interpatch dispersal.
2017. doi:\href{https://doi.org/10.1111/eva.12550}{10.1111/eva.12550}.

\hypertarget{ref-Levick2017}{}
5. Levick B, South A, Hastings IM. A Two-locus Model of The Evolution of
Insecticide Resistance to Inform and Optimise Public Health Insecticide
Deployment Strategies. PLOS Computational Biology. 2017;13:e1005327.

\hypertarget{ref-South2018}{}
6. South A, Hastings IM. Insecticide resistance evolution with mixtures
and sequences: A model-based explanation. Malaria Journal.
2018;17:1--20.
doi:\href{https://doi.org/10.1186/s12936-018-2203-y}{10.1186/s12936-018-2203-y}.

\hypertarget{ref-Kliot2012}{}
7. Kliot A, Ghanim M. Fitness costs associated with insecticide
resistance. Pest Management Science. 2012;68:1431--7.

\hypertarget{ref-Bourguet2000}{}
8. Bourguet D, Genissel A, Raymond M, Raymond AM. Insecticide Resistance
and Dominance Levels. Journal of Economic Entomology. 2000;93:1588--95.
doi:\href{https://doi.org/10.1603/0022-0493-93.6.1588}{10.1603/0022-0493-93.6.1588}.


\end{document}
